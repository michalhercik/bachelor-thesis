\chapter*{Závěr}
\addcontentsline{toc}{chapter}{Závěr}

Cílem této práce bylo vytvořit knihovnu pro web, pomocí které bude možné
vytvářet vizualizace několika sekundárních struktur RNA a porovnávat je mezi
sebou. V této práci jsme na začátek čtenáře krátce seznámili s RNA, její
funkcí\ref{rnauvod} a vizualizací. Následně jsme představili knihovnu napsanou
v jazyce Typescript, která pomocí jednoduchého rozhraní\ref{seznameni}
zpřístupňuje metody pro porovnání dostatečné podobných struktur\ref{porovnani}.
Knihovna nabízí plynulé vykreslování několika struktur najednou\ref{canvas} a
je jednoduše přístupná přes npm nebo GitHub\ref{seznameni}.

Ukázalo se, že vizualizace několika sekundárních struktur RNA je složitá.
Dokázali jsme vytvořit knihovnu pro vizualizaci, ale použité metody jsou vhodné
pouze pro dvě nebo tři struktury najednou, při větším množství se metody
neukázaly nápomocné, především protože vizualizace přestane být přehledná.
Navíc jsme se omezili na struktury, které jsou si dostatečně podobné.

V budoucnu by bylo užitečné knihovnu rozšířit o metody, které dokážou pracovat
s libovolnýma strukturami nebo se strukturami, které patří do stejné
klasifikace\cite{rfam}. Tím bychom se přiblížili k původnímu cíli této práce.
Dalším vylepšením by mohla být optimalizace kódu, aby byla knihovna schopná
pracovat plynule s větším množstvím struktur.
