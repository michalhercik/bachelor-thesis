
\chapter{Úvod do problematiky}

\section {Seznámení s biologickými pojmy}

\subsection{RNA} 

{\color{red}Rozdělit do více kapitol. Jedna z kapitol struktura RNA.}

RNA (zkratka z anglického ribonucleic acid) je biomolekula, která hraje
klíčovou roli v procesu přenosu genetické informace u všech živých organismů.
RNA se skládá z řetězce nukleotidů, které obsahují cukr ribózu, fosfátovou
skupinu a jednu z pěti dusíkatých bází (adenin, guanin, cytosin, uracil nebo
inosin). Existují různé typy RNA, jako jsou messenger RNA (mRNA), ribozomální
RNA (rRNA) a transfer RNA (tRNA), které mají každý svou specifickou funkci v
buňce.

RNA sekundární struktura se týká způsobu, jakým se molekula RNA skládá na sebe
díky vzniku bázových párů mezi komplementárními nukleotidy. Bázové párování se
děje mezi dusíkatými bázemi RNA nukleotidů, přičemž adenin (A) se páruje s
uracilem (U) a guanin (G) se páruje s cytosinem (C).

RNA sekundární struktura je důležitá, protože může ovlivnit to, jak RNA
molekula funguje. Například stem-loop struktura v mRNA molekule může ovlivnit
přístupnost mRNA k ribozomům, což je buněčný mechanismus zodpovědný za
překládání mRNA na proteiny.

\section{Vizualizace sekundárních RNA struktur} 

Pro reprezentaci sekundární RNA struktury se používají jak textové, tak
grafické způsoby. Pro nás jsou nejzajímavějsí ty grafické, ze kterých v této
části představíme tři nejpoužívanější - arc diagram, circular diagram a
radiate diagram. Obrázky ukázek diagramu v této části jsou získané za pomoci nástroje
VARNA\cite{Varna}.

V arc diagramu jsou nukleotidy zobrazeny na rovné čáře ve stejném pořadí jako v
sekvenci a bázové páry nukleotidů jsou spojeny obloukem.

\begin{figure}[H]
  \centering
  \includegraphics[width=140mm]{../img/kap01/arc.png}
  \caption{Ukázka arc diagramu}
\end{figure}

Circular diagram je velmi podobný. Nukleotidy neleží na rovné čáře, ale po
obvodu kruhu. Bázové páry jsou spojeny buď čárou nebo obloukem.

\begin{figure}[H]
  \centering
  \includegraphics[height=100mm]{../img/kap01/circular.png}
  \caption{Ukázka circular diagramu}
\end{figure}

Obě tyto reprezentace postrádají schopnost zachytit motivy sekundární
struktury, a proto se radiate diagram používá tam, kde je potřeba detailní
vizuální analýza motivů sekundární RNA struktury a její interakce. V radiate
diagramu jsou pozice nukleotidů voleny tak, aby bylo možné rozeznat motivy
sekundární struktury, jako jsou hairpins, bulges nebo vícevětvené smyčky.

\begin{figure}[H]
  \centering
  \includegraphics[height=100mm]{../img/kap01/radiate.png}
  \caption{Ukázka radiate diagramu}
\end{figure}

\section{Podobné projekty} {\color{red} chybi motivacni uvod, ktery by rekl, co chceme delat, aby bylo mozne pochopit podobne k cemu}

Rádi bychom čtenáře seznámili s některými nástroji, které jsou používáné pro
vizualizaci sekundárních RNA struktur. Většina z nich jsou programy s
uživatelským rozhraním a mohlo by se proto zdát zbytečné je zmiňovat nebo
porovnávat s naší knihovnou. Nicméně u níže zmíněných programů není duležité
řešení samotného uživatelského rozhraní, jako především druh zvolených metod
pro vizualizaci a následné porovnávání.

Z velkého množství existujících nástrojů, byla snaha vybrat takové,
které mají rozdílné přístupy a nabízí nejširší paletu funkcí.

\subsection{VARNA} 

VARNA (Visualization Applet for RNA) je nástroj pro automatické
kreslení, vizualizaci a anotaci sekundárních RNA struktur, navržený jako
doprovodný software pro webové servery a databáze.

VARNA implementuje algoritmy pro vykreslení všech tří výše zmíněných diagramů,
podporuje různé textové formáty pro vstup i výstup a je schopný exportovat
kresbu do rastrových nebo vektorových formátů. Umožňuje ruční úpravy a
strukturální anotace výsledku kresby a je považován za standard pro práci se
sekundárními strukturami RNA.

\begin{figure}[H]
  \centering
  \includegraphics[width=140mm]{../img/kap01/varna.png}
  \caption{Snimek nástroje Varna. Zobrazená struktura je d.5.b.A.madurae.}
\end{figure}

\subsection{RNAStructViz} 

RNAStructViz\cite{RnaStructViz} je grafický nástroj pro analýzu sekundárních
RNA struktur. Jeho předností je vizuální porovnání tří konfigurací v circular
arc diagramu. Doplněné zabudovaným prohlížečem CT-style\footnote{CT formát
souboru slouží k ukládání informace o sekvenci a bázových párů.} souboru a
prohlížečem radial diagramu podstruktury, která je přímo propojená s arc
diagram oknem skrze nástroj pro výběr zoom. Mezi další funkce patří vypočítání
číselných informací a možnost exportu obrázků a dat pro pozdější použití.

\begin{figure}[H]
  \centering
  \includegraphics[width=140mm]{../img/kap01/rnaStructViz.png}
  \caption{Snimek nástroje rnaStructViz, zobrazující tři struktury RNA.}
\end{figure}

\subsection{Forna} 

Forna\cite{Forna} (force-directed rna) nabízí webové rozhraní a server, který
umožňuje uživateli vložit sekundární RNA strukturu ve formátu dot-bracket a
zobrazí ji jako force-directed
graf\footnote{https://cs.brown.edu/people/rtamassi/gdhandbook/chapters/force-directed.pdf}.
Uživatel může následně upravit pozice přetažením myší a lze i upravovat přímo
strukturu. 

\begin{figure}[H]
  \centering
  \includegraphics[width=140mm]{../img/kap01/forna.png}
  \caption{Snimek nástroje Forna. Nalevo odvozená sekundární RNA
  struktura URS00000B9D9D\_471852 od struktury d.5.b.A.madurae napravo.}
\end{figure}

\subsection{R-chie} 

R-chie \cite{Rchie} je web server, který umí vygenerovat šest různých typů arc
diagramu. Vývoj tohoto nástroje byl se zaměřením především na složitější
struktury, které nelze hezky nakreslit v radial diagramu. R-chie umí
vygenerovat diagram pro porovnávání dvou sekundárních RNA struktur. Důležitým
cílem byla možnost generovat diagramy pro velké množství dat, proto také
nenabízí grafické rozhraní a s ním spojenou interakci se strukturami. 

Projekt také nabízí balíček napsaný v jazyce
R\footnote{https://www.r-project.org/} zvaný R4RNA, který umožňuje spuštění
programu lokálně a napříč operačním systémům.

\begin{figure}[H]
  \centering
  \includegraphics[width=140mm]{../img/kap01/rchie.jpeg}
  \caption{Výsledný arc diagram nástroje R-chie, zobrazující dvě sturktury.
  První struktura je nad horizontální čárou a druhá pod ní.}
\end{figure}

\subsection{Shrnutí existujících nástrojů}

Nástroje představené v této kapitole se soustředí především na prácí s circular
diagramem nebo arc diagramem, a právě pouze pro tyto diagramy nabízí nějaké
metody pro porovnávání omezeného množství sekundárních struktur RNA. Forna
podporuje pouze radial diagram, ale porovnávání dvou struktur, které sice jdou
zobrazit vedle sebe, už nijak neusnadňuje. 

Varna Podporuje všechny tři zmíněné diagrami, ale nelze ani zobrazit dvě
sekundární rna struktury vedle sebe. Velkou výhodou nástroje VARNA by byla
možnost použití na webu, ale k tomu používá Java Applets
\footnote{https://docs.oracle.com/javase/tutorial/deployment/applet/index.html},
které jsou od roku 2017 považované za zastaralé
\footnote{https://www.oracle.com/java/technologies/javase/9-deprecated-features.html}.

Ze zmíněných projektů je nejpodobnější tomu našemu R-chie, který se snaží
usnadnit porovnávání sekundárních RNA struktur a nabízí i knihovnu napsanou v
jazyce R. Liší se pak v samotném přístupu, protože jejich rozhraní generuje
pouze statické circular nebo arc diagramy.

\section{Kreslení grafů na základě šablony}

Níže jsou zmíněné dva projekty, které úzce souvisí s naší knihovnou, protože
produkují data ve formátu, se kterým pracuje naše knihovna a metody
použité ke generovaní takových dat jsou klíčové pro naší knihovnu.

\subsection{TRAVeLer} 

Traveler\cite{Traveler2017} je nástroj pro vizualizaci cílové sekundární
struktury, využívající existující rozložení dostatečně podobné RNA struktury
jako vzor. Traveler je založený na algoritmu, který konvertuje cílovou a
vzorovou strukturu do odpovídající stromové reprezentace a využije stromovou
editační vzdálenost společně s modifikací rozložení k přetvoření vzorové
struktury do cílové. Traveler přijme na vstupu sekundární strukturu a vzor
rozložení a na výstupu dá rozložení cílové struktury. Je to tedy command-line
open source nástroj schopný rychle generovat rozložení i pro největší RNA
struktury za poskytnutí dostatečně podobného rozložení.

Do vzniku Traveleru neexistoval žádný nástroj, který by dokázal velké struktury
vizualizovat ve standardni notaci, se kterou jsou biologové naučení pracovat a
porovnávat struktury napříč druhům.

\subsection{R2DT} 

R2DT\cite{R2DT2021} je metoda pro predikci a vizualizaci široké škály
sekundárních RNA struktur ve radial diagramu. R2DT je postaveno na knihovně se
3 647 vzory reprezentujícími většinu známých RNA struktur. R2DT se používá na
ncRNA\footnote{RNA, která se nepřekládá do proteinů} (non-coding RNA)
sekvencích z RNAcentral\footnote{https://rnacentral.org/} databáze a vytvořila
více než 27 miliónů diagramů\footnote{Číslo je aktuální k datu 11.4. 2023},
čímž tvoří největší světovou sadu dat s 2D RNA strukturami. Pro vizualizaci
neboli 2D rozložení používá R2DT právě výše zmíněný nástroj Traveler.
