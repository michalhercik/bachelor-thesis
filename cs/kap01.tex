
\chapter{Úvod do problematiky}

\section {Seznámení s biologickými pojmy}

\subsection{Nukleotid}

\subsection{Sekundární struktura RNA}
  RNA neboli kyselina ribonukleová je nukleová kyselina, skládající se z
  řetězce tvořeného vzájemně propojenými nukleotidy. Ty jsou tvořeny z některé
  z bází adenin, uracil, guanin a cytosin. Je obvykle jednovláknová, někdy i
  dvouvláknová. Její přesná funkce se liší v závislosti na typu, ale slouží k
  přenosu genetické informace z jádra buňky na místo syntézy bílkovin, a také
  se procesu syntézy přímo účastní.

  Primární struktura RNA je dána přesným pořadím nukleotidů v řetězci.
  Sekundární struktura RNA má podobu jednoduchého stočeného řetězce. V
  některých druzích RNA lze nalézt i zdvojené úseky řetězce. Terciální
  struktura RNA pak popisuje prostorové rozložení.

\section {Používané datové zdroje}

\subsection{Datové fromáty}

\subsubsection*{JSON}
  JSON (javascriptový objektový zápis) je datový formát sloužící k ukládání dat
  organizovaných v polích nebo objektech. Navzdory názvu je na programovacím
  jazyce nezávislý. Skládá se z dvojice klíč -- hodnota. Hodnota je libovolný
  podporovaný datový typ (např.: boolean, číslo, string, pole, objekt). Níže je
  úkázka JSON formátu.

\begin{code}
  { 
    "basePairs": [
      {
        "basePairType": "canonical",
        "classes": [
          "bp-line"
        ],
        "residueIndex1": 2,
        "residueIndex2": 118
      }
    ]    
  }
\end{code}

\section{Seznámení s typama vizualizace sekundárních RNA struktur}

\subsection{Standardní layout}

\subsection {Arc diagram}

\subsection {radial layout}

\section{Podobné projekty}
Rádi bychom čtenáře seznámili s některými nástroji, které jsou používáné pro
vizualizaci sekundárních RNA struktur. Většina z nich je program a mohlo by se
proto zdát zbytečné je zmiňovat nebo porovnávat s naší knihovnou. Nicméně u
níže zmíněných programů není duležité řešení samotného grafického rozhraní,
jako především druh zvolených metod pro vizualizaci a následné porovnávání.

Nejsou zde zmíněné všechny existující nástroje, ale byla snaha vybrat takové,
které mají rozdílné přístupy a nabízí nejširší paletu funkcí.

\subsection{VARNA}
  VARNA\cite{Varna} (Visualization Applet for RNA) je nástroj pro automatické kreslení,
  vizualizaci a anotaci sekundárních RNA struktur, navržený jako doprovodný
  software pro webové servery a databáze.

  VARNA implementuje čtyři kreslící algoritmy, podporuje formáty dbn, ct,
  bpseq a RNAML pro vstup i výstup a je schopné exportovat kresbu do rastrových
  nebo vektorových formátů. Umožňuje ruční úpravy a strukturální anotace
  výsledku kresby a je považována za standard pro práci se sekundárníma RNA
  strukturama.

  Kromě samotného faktu, že VARNA je program se liší od náší knihovny i v
  dalších bodech. Zatímco VARNA je rozumný nástroj pro anotaci a vykreslení
  výše zmíněných textových formátů různými způsoby, naše knihovna se snaží
  poskytnout pouze metody pro porovnávání sekundárních RNA struktur s již daným
  rozložením nukletidů v prostoru. VARNA navíc pro implementaci na web používá
  Java Applets
  \footnote{https://docs.oracle.com/javase/tutorial/deployment/applet/index.html},
  které jsou od roku 2017 považované za zastaralé
  \footnote{https://www.oracle.com/java/technologies/javase/9-deprecated-features.html},
  tím pádem je nelze považovat za vhodnou variantu pro použití na webu.

\subsection{RNAStructViz}
  RNAStructViz\cite{RnaStructViz} je grafický nástroj pro analýzu sekundárních
  RNA struktur. Jeho předností je vizuální porovnání tří konfigurací v
  kompaktním a standartizovaným {\color{red} TODO: přeložit???} circular arc
  diagramu. Doplněné zabudovaným prohlížečem {\color{red} TODO: přeložit???}
  CT-style souboru a prohližeče {\color{red} TODO: přeložit???} radial layout
  podstruktury, která je přímo propojená s {\color{red} TODO: přeložit???} arc
  diagram oknem skrze nástroj pro výběr zoom. Mezi další funkce patří
  vypočítání číselných informací a možnost exportu obrázků a dat pro pozdější
  použití.

  Hlavním rozdílem v RNAStructViz kromě samotného faktu, že se
  jedná o program a ne knihovnu, vidíme zvolené metody porovnání.
  Zatímco naše knihovna používá standartní rozložení bez omezení
  na počet struktur, tak RNAStructViz používá {\color{red} TODO:
  přeložit?} radial layout s maximální podporou zobrazení tří
  struktur najednou.

\subsection{Forna}
  Forna\cite{Forna} (force-directed rna) nabízí webové rozhraní a server, který
  umožňuje uživateli vložit sekundární RNA strukturu ve formátu dot-bracket a
  zobrazí ji jako force-directed graf. Uživatel může následně upravit pozice
  přetažením myší a lze i upravovat přímo strukturu. Forna umožňuje zobrazení
  libovolného množství struktur vedle sebe, ale nenabízí metody pro porovnávání
  mezi sebou.

\subsection{R-chie}
  R-chie \cite{Rchie} je web server, který umí vygenerovat šest různých typů
  {\color{red} TODO: přeložit?} arc diagramu. Vývoj tohoto nástroje byl se
  zaměřením především na složitější struktury, které nelze hezky nakreslit v
  rovinném diagramu. R-chie umí vygenerovat diagram pro porovnávání dvou
  sekundárních RNA struktur. Důležitým cílem bylo možnost generovat diagramy
  pro velké množství dat, proto také nenabízí grafické rozhraní a s ním
  spojenou interakci se strukturama. 

  Kromě metody vizualizace se liší od naší knihovny právě ve zmíněné snaze
  zpracovat velké množství dat bez interakce, zatím co naše knihovna se
  soustředí na interakci se sekundárníma RNA strukturama. Projekt také nabízí
  balíček napsaný v jazyce R\footnote{https://www.r-project.org/} zvaný R4RNA,
  který umožňuje spuštění programu lokálně a napříč operačním systémům.

\section{Příbuzné projekty}
Níže jsou zmíněné dva projekty, které úzce souvisí s naší knihovnou, protože
produkují vstupní data ve formátu, se kterým pracuje naše knihovna a metody
použité v generovaní takových dat jsou klíčové pro naší knihovnu.

\subsection{TRAVeLer}
  Traveler\cite{Traveler2017} je nástroj pro vizualizaci cílové sekundární
  struktury, využívající existující rozložení dostatečně podobné RNA struktury
  jako vzor. Traveler je založený na algoritmu, který konvertuje cílovou a
  vzorovou strukturu do odpovídající stromové reprezentace a využije stromovou
  editační vzdálenost společně s modifikace rozložení k přetvoření vzorové
  struktury do cílové. Traveler přijme na vstupu sekundární strukturu a
  vzor rozložení a na výstupu dá rozložeí cílové struktury. Je to tedy
  command-line open source nástroj schopný rychle generovat rozložení i pro
  největší RNA struktury za poskytnutí dostatečně podobného rozložení.

\subsection{R2DT}
  R2DT\cite{R2DT2021} je metoda pro predikci a vizualizaci široké škály
  sekundárních RNA struktur ve standardním rozložení. R2DT je postaveno na
  knihovně se 3 647 vzory reprezentujícími většinu známých RNA struktur. R2DT
  se používá na ncRNA sekvence z RNAcentral\footnote{https://rnacentral.org/}
  databáze a vytvořil více než 13 miliónů diagramů, čímž tvoří největší
  světovou sadu dat s 2D RNA strukturami. Pro vizualizaci neboli 2D rozložení
  používá právě výše zmíněný nástroj Traveler.
