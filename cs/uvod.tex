\chapter*{Úvod}
\addcontentsline{toc}{chapter}{Úvod}

RNA má více funkcí než pouze přenos genetické informace. Zkoumání struktur RNA
napříč různými živočišnými druhy a sledování podobností, které mohou hrát
důležitou roli v její funkci, může vést k lepšímu pochopení
evoluce\cite{kyticky} nebo některých nemocí\cite{disease}, speciálně se zkoumá
například v souvislosti s rakovinou\cite{cancer, cancer2} nebo neurologickými
onemocněními\cite{neuro, neuro2}.

Funkce RNA je odvozená od její 3D struktury (terciární struktury) a ačkoliv
její predikce je aktivně zkoumané téma\cite{3DStructure1, 3DStructure2}, je
stále těžké jí získat. Mnohem více známé jsou sekundární struktury. Ty o
prostorovém rozložení říkají pouze to, že spárované nukleotidy jsou blízko
sebe, ale přesto nabízí dobrý popis.

Pro analýzu sekundárních struktur RNA existuje mnoho nástrojů. Nástroje se
typicky soustředí na možnosti anotace, editace nebo porovnávání omezeného
množství sekundárních struktur RNA a neexistuje nástroj, který by umožňoval
interaktivně porovnávat mezi sebou libovolné množství sekundárních struktur
RNA. Většina nástrojů navíc nenabízí možnost integrace do jiných programů,
například do webových databází sekundárních struktur.

Z těchto důvodů představujeme knihovnu napsanou v jazyce Typescript, která
umožňuje jednoduše vizualizovat sekundární struktury a nabízí metody pro
porovnávaní podobných struktur, které můžou hrát klíčovou roli k pochopení
biologických mechanismů.
