%%% Hlavní soubor. Zde se definují základní parametry a odkazuje se na ostatní části. %%%

%% Verze pro jednostranný tisk:
% Okraje: levý 40mm, pravý 25mm, horní a dolní 25mm
% (ale pozor, LaTeX si sám přidává 1in)
\documentclass[12pt,a4paper]{report}
\setlength\textwidth{145mm}
\setlength\textheight{247mm}
\setlength\oddsidemargin{15mm}
\setlength\evensidemargin{15mm}
\setlength\topmargin{0mm}
\setlength\headsep{0mm}
\setlength\headheight{0mm}
% \openright zařídí, aby následující text začínal na pravé straně knihy
\let\openright=\clearpage

%% Pokud tiskneme oboustranně:
% \documentclass[12pt,a4paper,twoside,openright]{report}
% \setlength\textwidth{145mm}
% \setlength\textheight{247mm}
% \setlength\oddsidemargin{14.2mm}
% \setlength\evensidemargin{0mm}
% \setlength\topmargin{0mm}
% \setlength\headsep{0mm}
% \setlength\headheight{0mm}
% \let\openright=\cleardoublepage

%% Vytváříme PDF/A-2u
\usepackage[a-2u]{pdfx}

%% Přepneme na českou sazbu a fonty Latin Modern
\usepackage[czech]{babel}
\usepackage{lmodern}
\usepackage[T1]{fontenc}
\usepackage{textcomp}

%% Použité kódování znaků: obvykle latin2, cp1250 nebo utf8:
\usepackage[utf8]{inputenc}

%%% Další užitečné balíčky (jsou součástí běžných distribucí LaTeXu)
\usepackage{amsmath}        % rozšíření pro sazbu matematiky
\usepackage{amsfonts}       % matematické fonty
\usepackage{amsthm}         % sazba vět, definic apod.
\usepackage{bbding}         % balíček s nejrůznějšími symboly
			    % (čtverečky, hvězdičky, tužtičky, nůžtičky, ...)
\usepackage{bm}             % tučné symboly (příkaz \bm)
\usepackage{graphicx}       % vkládání obrázků
\usepackage{fancyvrb}       % vylepšené prostředí pro strojové písmo
\usepackage{indentfirst}    % zavede odsazení 1. odstavce kapitoly
\usepackage[square,numbers]{natbib}         % zajištuje možnost odkazovat na literaturu
			    % stylem AUTOR (ROK), resp. AUTOR [ČÍSLO]
\usepackage[nottoc]{tocbibind} % zajistí přidání seznamu literatury,
                            % obrázků a tabulek do obsahu
\usepackage{icomma}         % inteligetní čárka v matematickém módu
\usepackage{dcolumn}        % lepší zarovnání sloupců v tabulkách
\usepackage{booktabs}       % lepší vodorovné linky v tabulkách
\usepackage{paralist}       % lepší enumerate a itemize
\usepackage{xcolor}         % barevná sazba
\usepackage{float}	    % nevložit figure až na novou stránku
\usepackage{listings}
\usepackage{pdfpages}

\lstset{
frame=single,
basicstyle=\footnotesize\ttfamily,
extendedchars=true,
literate={é}{{\'e}}1
           {č}{{\v{c}}}1
           {ľ}{{\v{l}}}1
           {ť}{{\v{t}}}1
           {ý}{{\'y}}1
           {ě}{{\v{e}}}1
           {ř}{{\v{r}}}1
           {š}{{\v{s}}}1
           {ž}{{\v{z}}}1
           {á}{{\'a}}1
           {í}{{\'i}}1
           {ó}{{\'o}}1
           {ň}{{\v{n}}}1
           {ď}{{\v{d}}}1
           {ú}{{\'u}}1
           {ů}{{\r{u}}}1
           {ĺ}{{\v{l}}}1
}

\makeatletter
\renewcommand{\@makefntext}[1]{%
  \setlength{\parindent}{0pt}%
  \begin{list}{}{%
    \setlength{\labelwidth}{1.5em}% <===================================
    \setlength{\leftmargin}{\labelwidth}%
    \setlength{\labelsep}{3pt}%
    \setlength{\itemsep}{0pt}%
    \setlength{\parsep}{0pt}%
    \setlength{\topsep}{0pt}%
    \setlength{\rightmargin}{0.2\textwidth}%
    \footnotesize}%
  \item[\@makefnmark\hfil]#1%
  \end{list}%
}
\makeatother

%%% Údaje o práci

% Název práce v jazyce práce (přesně podle zadání)
\def\NazevPrace{Webový plugin pro vizualizaci sady sekundárních struktur RNA}

% Název práce v angličtině
\def\NazevPraceEN{Web plugin for multiple RNA secondary structure visualization}

% Jméno autora
\def\AutorPrace{Michal Hercík}

% Rok odevzdání
\def\RokOdevzdani{2023}

% Název katedry nebo ústavu, kde byla práce oficiálně zadána
% (dle Organizační struktury MFF UK, případně plný název pracoviště mimo MFF)
\def\Katedra{Katedra softwarového inženýrství}
\def\KatedraEN{Department of Software Engineering}

% Jedná se o katedru (department) nebo o ústav (institute)?
\def\TypPracoviste{Katedra}
\def\TypPracovisteEN{Department}

% Vedoucí práce: Jméno a příjmení s~tituly
\def\Vedouci{doc. RNDr. David Hoksza, Ph.D.}

% Pracoviště vedoucího (opět dle Organizační struktury MFF)
\def\KatedraVedouciho{Katedra softwarového inženýrství}
\def\KatedraVedoucihoEN{Department of Software Engineering}

% Studijní program a obor
\def\StudijniProgram{Informatika}
\def\StudijniObor{Programování a vývoj software}

% Nepovinné poděkování (vedoucímu práce, konzultantovi, tomu, kdo
% zapůjčil software, literaturu apod.)
\def\Podekovani{%
Rád bych poděkoval všem, díky kterým pro mě bylo možné tuto práci napsat.
Speciálně potom svému vedoucímu práce doc. RNDr. Davidu Hokszovi, Ph.D. za
veškeré konzultace a odbornou pomoc. 
}

% Abstrakt (doporučený rozsah cca 80-200 slov; nejedná se o zadání práce)
\def\Abstrakt{%
Zkoumání RNA je důležité pro lepší pochopení evoluce nebo některých onemocnění.
V této práci představujeme knihovnu napsanou v jazyce Typescript, která nabízí
metody pro vizuální analýzu více sekundárních struktur RNA, nejlépe ale metody
fungují pří práci s dvěmi nebo třemi strukturami. Pro reprezentaci používáme
radial diagram, který je vygenerovaný na základě vzorové struktury. Metody
použité v knihovně pro analýzu využívají způsobu generování radial diagramu a
to tak, že diagramy vygenerované ze stejné vzorové struktury na sebe mapuje
právě pomocí vzorové struktury, díky tomu lze sledovat rozdíly a podobnosti
struktur.
}
\def\AbstraktEN{%
The study of RNA is important to better understand evolution or some
diseases. In this thesis, we present a library written in Typescript that
offers methods for visual analysis of multiple RNA secondary structures, but
preferably work best with two or three structures. For the representation we
use radial diagram, which is generated by template-based method. Methods used
in the library for analysis use the template-based generation of the radial
diagram by mapping diagrams generated from the same template to each other via
the template, so that differences and similarities of structures can be
observed.
}

% 3 až 5 klíčových slov (doporučeno), každé uzavřeno ve složených závorkách
\def\KlicovaSlova{%
{bioinformatika} {RNA} {sekundární struktura} {web} {plugin}
}
\def\KlicovaSlovaEN{%
{bioinformatics} {RNA} {secondary structure} {web} {plugin}
}

%% Balíček hyperref, kterým jdou vyrábět klikací odkazy v PDF,
%% ale hlavně ho používáme k uložení metadat do PDF (včetně obsahu).
%% Většinu nastavítek přednastaví balíček pdfx.
\hypersetup{unicode}
\hypersetup{breaklinks=true}

%% Definice různých užitečných maker (viz popis uvnitř souboru)
\include{makra}

%% Titulní strana a různé povinné informační strany
\begin{document}
% \include{desky}
% \includepdf[pages=-]{../pdf/zadani.pdf}
\include{titulka}

%%% Strana s automaticky generovaným obsahem bakalářské práce

\tableofcontents

%%% Jednotlivé kapitoly práce jsou pro přehlednost uloženy v samostatných souborech
\chapter*{Úvod}
\addcontentsline{toc}{chapter}{Úvod}

TODO: uvod


\chapter{Úvod do problematiky} \label{rnauvod}

\section{RNA}

\footnote{Celá tato kapitola byla vygenerovaná pomocí ChatGPT dostupné v dubnu
2023 na adrese https://chat.openai.com/ a upravena.} RNA (Ribonukleová
kyselina) je typ nukleové kyseliny, která je nezbytná pro různé biologické
procesy. Skládá se z dlouhého řetězce nukleotidů obsahujících cukr, fosfátovou
skupinu a dusíkatou bázi. Čtyři dusíkaté báze v RNA jsou adenin, guanin,
cytosin a uracil. RNA je transkribována z DNA a může být nalezena všude v
živých buňkách.

\subsection{Funkce}

RNA má několik důležitých funkcí v buňkách. Jednou z jejích hlavních funkcí je
fungovat jako posel mezi DNA a ribozomy během syntézy proteinů. Messenger RNA
(mRNA) nese genetickou informaci z DNA k ribozomům, které tuto informaci
používají k vytváření proteinů. Další typy RNA, jako tRNA a rRNA, hrají také
důležité role v syntéze proteinů a patří do skupiny zvané nekódující RNA
(ncRNA).

ncRNA je druh RNA, která nekóduje proteiny, ale místo toho hraje regulační role
v buňkách. ncRNA lze rozdělit do dvou širokých kategorií: malé ncRNA, které
jsou menší než 200 nukleotidů, a dlouhé ncRNA, které jsou delší než 200
nukleotidů. Malé ncRNA, jako jsou mikroRNA a siRNA, se účastní posttranskripční
regulace genů, zatímco dlouhé ncRNA se účastní různých buněčných procesů,
včetně regulace exprese genů, remodelace chromatinu a inaktivace chromozomu X.

\subsection{Sekundární struktura RNA}

RNA molekuly se skládají do specifických tvarů díky komplementárnímu párování
nukleotidů uvnitř molekuly. To vede k vytvoření sekundárních struktur, které
jsou stabilizovány vodíkovými vazbami mezi bázemi. Sekundární struktury RNA
hrají důležité role v různých buněčných procesech, jako je syntéza proteinů a
regulace genů.

\subsubsection{Motivy sekundární struktury RNA}

Sekundární struktury RNA mohou být složeny z různých strukturálních motivů,
jako jsou hairpin loop, bulge a multibranch loop.

\paragraph{Hairpin loop}

Jedním z nejčastějších typů sekundární struktury je hairpin loop\ref{hairpin}, který se
skládá z jednořetězcové oblasti RNA, která se ohýbá zpět na sebe a vytváří kmen
(dvouřetězcovou oblast) a smyčku (jednořetězcovou oblast). Hairpin smyčky se
často nacházejí na koncích molekul RNA a mohou hrát důležité role v stabilitě a
zpracování RNA.

\begin{figure}[H]
  \centering
  \includegraphics[height=40mm]{../img/kap01/rna/hairpin.png}
  \caption[Ukázka motivu hairpin loop]{Ukázka motivu hairpin loop.}
  \label{hairpin}
\end{figure}

\paragraph{Bulge}

Dalším častým typem sekundární struktury je bulge\ref{bulge}, který zahrnuje
jeden nukleotid, který není spárovaný s komplementárním nukleotidem v kmenové
oblasti. Bulges mohou narušit sekundární strukturu RNA, ale také mohou být
důležité v interakci mezi proteinem a RNA.

\begin{figure}[H]
  \centering
  \includegraphics[width=20mm]{../img/kap01/rna/bulge.png}
  \caption[Ukázka motivu bulge]{Ukázka motivu bulge.}
  \label{bulge}
\end{figure}

\paragraph{Multibranch loop}

\sloppy

Multibranch loop\ref{multibranch} je složitější sekundární struktura, která
zahrnuje tvorbu více stonků a smyček. Tyto struktury mohou být důležité pro
skládání a funkci RNA a často se nacházejí v molekulách RNA s katalytickou
aktivitou, jako jsou ribozymy.

\fussy

\begin{figure}[H]
  \centering
  \includegraphics[height=40mm]{../img/kap01/rna/multibranch.png}
  \caption[Ukázka motivu multibranch loop]{Ukázka motivu multibranch loop.}
  \label{multibranch}
\end{figure}

Celkově je pochopení sekundární struktury RNA zásadní pro pochopení její
biologické funkce a toho, jak interaguje s jinými molekulami v buňce.

\section{Vizualizace sekundárních RNA struktur} 

Pro reprezentaci sekundární RNA struktury se používají jak textové, tak
grafické způsoby. Pro analýzu jsou vhodnější ty grafické, protože dokáží mnohem
lépe odhalit stavbu struktury.

V této části představíme tři nejpoužívanější grafické reprezentace - arc
diagram, circular diagram a radiate diagram. Obrázky ukázek diagramu v této
části jsou získané za pomoci nástroje VARNA\cite{Varna}.

V arc diagramu\ref{arc} jsou nukleotidy zobrazeny na rovné čáře ve stejném pořadí jako v
sekvenci a bázové páry nukleotidů jsou spojeny obloukem.

\begin{figure}[H]
  \centering
  \includegraphics[width=140mm]{../img/kap01/diagrams/arc.png}
  \caption[Ukázka arc diagramu]{Ukázka arc diagramu.}
  \label{arc}
\end{figure}

Circular diagram\ref{circ} je velmi podobný. Nukleotidy neleží na rovné čáře, ale po
obvodu kruhu. Bázové páry jsou spojeny buď čárou nebo obloukem.

\begin{figure}[H]
  \centering
  \includegraphics[height=100mm]{../img/kap01/diagrams/circular.png}
  \caption[Ukázka circular diagramu]{Ukázka circular diagramu.}
  \label{circ}
\end{figure}

V radiate diagramu\ref{radiate} jsou pozice nukleotidů voleny tak, aby bylo možné rozeznat
motivy sekundární struktury, jako jsou hairpins, bulges nebo multibranch loops. 

\begin{figure}[H]
  \centering
  \includegraphics[height=100mm]{../img/kap01/diagrams/radiate.png}
  \caption[Ukázka radiate diagramu]{Ukázka radiate diagramu.}
  \label{radiate}
\end{figure}

Právě schopnost zachytit zmíněné motivy arc i circular diagramy postrádají, a
proto se radiate diagram používá tam, kde je potřeba detailní vizuální analýza
motivů sekundární RNA struktury a její interakce. 


\section{Podobné projekty} \label{projekty}

Rádi bychom čtenáře seznámili s některými nástroji, které jsou používané pro
vizualizaci sekundárních RNA struktur. Většina z nich jsou programy s
uživatelským rozhraním a mohlo by se proto zdát zbytečné je zmiňovat nebo
porovnávat s naší knihovnou. Nicméně u níže zmíněných programů není důležité
řešení samotného uživatelského rozhraní, jako především druh zvolených metod
pro vizualizaci a následné porovnávání.

Z velkého množství existujících nástrojů byla snaha vybrat takové, které mají
rozdílné přístupy a nabízí nejširší paletu funkcí.

\subsection{VARNA} 

VARNA\ref{var} (Visualization Applet for RNA) je nástroj pro automatické
kreslení, vizualizaci a anotaci sekundárních RNA struktur, navržený jako
doprovodný software pro webové servery a databáze.

VARNA implementuje algoritmy pro vykreslení všech tří výše zmíněných diagramů,
podporuje různé textové formáty pro vstup i výstup a je schopný exportovat
kresbu do rastrových nebo vektorových formátů. Umožňuje ruční úpravy a
strukturální anotace výsledku kresby a je považován za standard pro práci se
sekundárními strukturami RNA.

\begin{figure}[H]
  \centering
  \includegraphics[width=140mm]{../img/kap01/tools/varna.png}
  \caption[Snímek nástroje Varna]{Snímek nástroje Varna.}
  \label{var}
\end{figure}

\subsection{RNAStructViz} 

RNAStructViz\cite{RnaStructViz}\ref{structviz} je grafický nástroj pro analýzu sekundárních
RNA struktur. Jeho předností je vizuální porovnání tří konfigurací v circular
arc diagramu. Doplněné zabudovaným prohlížečem CT-style\footnote{CT formát
souboru slouží k ukládání informace o sekvenci a bázových párů.} souboru a
prohlížečem radial diagramu podstruktury, která je přímo propojená s arc
diagram oknem skrze nástroj pro výběr zoom. Mezi další funkce patří vypočítání
číselných informací a možnost exportu obrázků a dat pro pozdější použití.

\begin{figure}[H]
  \centering
  \includegraphics[width=140mm]{../img/kap01/tools/rnaStructViz.png}
  \caption[Snímek nástroje rnaStructViz]{Snímek nástroje rnaStructViz,
  zobrazující tři sekundární struktury RNA.\protect\footnotemark}
  \label{structviz}
\end{figure}

\footnotetext{https://github.com/gtDMMB/RNAStructViz/wiki/ArcDiagrams}

\subsection{Forna} 

Forna\cite{Forna}\ref{fornascreen} (force-directed rna) nabízí webové rozhraní
a server, který umožňuje uživateli vložit sekundární RNA strukturu ve formátu
dot-bracket a zobrazí ji jako force-directed
graf\footnote{https://cs.brown.edu/people/rtamassi/gdhandbook/chapters/force-directed.pdf}.
Uživatel může následně upravit pozice přetažením myší a lze i upravovat přímo
strukturu. 

\begin{figure}[H]
  \centering
  \includegraphics[width=140mm]{../img/kap01/tools/forna.png}
  \caption[Snímek nástroje Forna]{Snímek nástroje Forna se dvěmi sekundárními
  strukturami.}
  \label{fornascreen}
\end{figure}

\subsection{R-chie} 

R-chie \cite{Rchie}\ref{rchiescreen} je web server, který umí vygenerovat šest
různých typů arc diagramu. Vývoj tohoto nástroje byl se zaměřením především na
složitější struktury, které nelze hezky nakreslit v radial diagramu. R-chie umí
vygenerovat diagram pro porovnávání dvou sekundárních struktur RNA. Důležitým
cílem byla možnost generovat diagramy pro velké množství dat, proto také
nenabízí grafické rozhraní a s ním spojenou interakci se strukturami. 

Projekt také nabízí balíček napsaný v jazyce
R\footnote{https://www.r-project.org/} zvaný R4RNA, který umožňuje spuštění
programu lokálně a napříč operačním systémům.

\begin{figure}[H]
  \centering
  \includegraphics[width=140mm]{../img/kap01/tools/rchie.jpeg}
  \caption[Výsledný arc diagram nástroje R-chie]{Výsledný arc diagram nástroje
  R-chie, zobrazující dvě sturktury. První struktura je nad horizontální čárou
  a druhá pod ní.\protect\footnotemark}
  \label{rchiescreen}
\end{figure}

\footnotetext{https://www.e-rna.org/r-chie/}

\subsection{Shrnutí existujících nástrojů}

Nástroje představené v této kapitole se soustředí především na práci s circular
diagramem nebo arc diagramem, a právě pouze pro tyto diagramy nabízí nějaké
metody pro porovnávání omezeného množství sekundárních struktur RNA. Forna
podporuje pouze radial diagram, ale porovnávání dvou struktur, které sice jdou
zobrazit vedle sebe, už nijak neusnadňuje. 

Varna Podporuje všechny tři zmíněné diagramy, ale nelze ani zobrazit dvě
sekundární struktury RNA vedle sebe. Velkou výhodou nástroje VARNA by byla
možnost použití na webu, ale k tomu používá Java Applets
\footnote{https://docs.oracle.com/javase/tutorial/deployment/applet/index.html},
které jsou od roku 2017 považované za zastaralé
\footnote{https://www.oracle.com/java/technologies/javase/9-deprecated-features.html}.

Ze zmíněných projektů je nejpodobnější tomu našemu R-chie, který se snaží
usnadnit porovnávání sekundárních RNA struktur a nabízí i knihovnu napsanou v
jazyce R. Liší se pak v samotném přístupu, protože jejich rozhraní generuje
pouze statické circular nebo arc diagramy.

\section{Kreslení grafů na základě šablony}

Níže jsou zmíněné dva projekty, které úzce souvisí s naší knihovnou, protože
produkují data ve formátu, se kterým pracuje naše knihovna a metody
použité ke generovaní takových dat jsou klíčové pro naší knihovnu.

\subsection{TRAVeLer} \label{traveler}

Traveler\cite{Traveler2017} je nástroj pro vizualizaci cílové sekundární
struktury, využívající existující rozložení dostatečně podobné RNA struktury
jako vzor. Traveler je založený na algoritmu, který konvertuje cílovou a
vzorovou strukturu do odpovídající stromové reprezentace a využije stromovou
editační vzdálenost společně s modifikací rozložení k přetvoření vzorové
struktury do cílové\ref{editshow}\ref{travelerdemo}. Traveler přijme na vstupu
sekundární strukturu a vzor rozložení a na výstupu dá rozložení cílové
struktury. Je to tedy command-line open source nástroj schopný rychle generovat
rozložení i pro největší RNA struktury za poskytnutí dostatečně podobného
rozložení.

Do vzniku Traveleru neexistoval žádný nástroj, který by dokázal velké struktury
vizualizovat ve standardní notaci, se kterou jsou biologové naučení pracovat a
porovnávat struktury napříč druhům.

\begin{figure}[H]
  \centering
  \includegraphics[width=100mm]{../img/kap01/traveler/editation.png}
  \caption[Ilustrace jednoduchých editací vzorové struktury]{Ilustrace
  jednoduchých editací vzorové struktury.\cite{Traveler2017}}
  \label{editshow}
\end{figure}

\begin{figure}[H]
  \centering
  \includegraphics[width=140mm]{../img/kap01/traveler/demo.png}
  \caption[Ukázka výsledku nástroje Traveler]{Rozložení malé podjednotky
  lidské ribozomální RNA vytvořené různými nástroji. (a) Layout ve formě, na
  kterou je zvyklá biologická komunita. (b) Rozložení generované Travelerem
  pomocí ovocné mušky jako šablony. (c) Rozložení generované nástrojem VARNA
  (verze 3-93). (d) Rozložení generované nástrojem
  RNAplot\cite{RNAplot}\cite{Traveler2017}.}
  \label{travelerdemo}
\end{figure}

\subsection{R2DT} \label{r2dt}

R2DT\cite{R2DT2021} je metoda pro predikci a vizualizaci široké škály
sekundárních RNA struktur ve radial diagramu. R2DT je postaveno na knihovně se
3 647 vzory reprezentujícími většinu známých RNA struktur. R2DT se používá na
ncRNA sekvencích z RNAcentral\footnote{https://rnacentral.org/} databáze a
vytvořila více než 27 miliónů diagramů\footnote{Číslo je aktuální k datu 11.4.
2023}, čímž tvoří největší světovou sadu dat s 2D RNA strukturami. Pro
vizualizaci neboli 2D rozložení používá R2DT právě výše zmíněný nástroj
Traveler. Ostatní kroky jsou znázorněné v
diagramu\ref{pipe}\footnote{https://github.com/rnacentral/r2dt}.

\begin{figure}[H]
  \centering
  \includegraphics[width=140mm]{../img/kap01/traveler/pipe.png}
  \caption[Jednotlivé kroky nástroje R2DT]{Jednotlivé kroky nástroje R2DT.}
  \label{pipe}
\end{figure}


\chapter{Metody vizualizace a porovnání}

Cílem této knihovny není pouze vizualizovat sekundární strukturu RNA, ale také
usnadnit analýzu rozdílů a podobností mezi více strukturami RNA. Proto jsme se
zaměřili na práci s radial diagramy, které nejlépe zobrazujou motivy struktury
a zároveň jsou nejpřirozenější reprezentací.

Kromě toho jsme viděli potenciál generování rozložení na základě vzorové
struktury, jak to dělá nástroj Traveler. Výstupem Traveleru je soubor ve
formátu JSON, který obsahuje informace o vzoru každého nukleotidu a provedených
úpravách - přidání, odebrání, přejmenování a přesunutí nukleotidu.

Například pokud bychom použili nástroj Varna na zobrazení následujících dvou
struktur musíme všechny podobnosti vypozorovat sami. 

\begin{figure}[H]
  \centering
  \includegraphics[width=140mm]{../img/kap02/intro/varna.png}
  \caption{Dvě sekundární RNA struktury s RNAcentral ID A) URS00006E712C, B)
  URS0000AB09C9.}
\end{figure}

Sice není těžké vypozorovat některé podobné motivy, ale není na první pohled
jasná podobnost sekvence.

Na generování obou struktur používá nástroj Traveler stejnou vzorovou
strukturu, která je v tomto konkrétním případě podobná obou odvozeným
strukturám. Podobnost odvozených struktur ke vzorové je až na výjimky běžná, a
proto jsme se rozhodli tuto podobnost předpokládat. Následující obrázek je
vizualizace stejných struktur, za použití nástroje Traveler spolu se vzorovou
strukturou.

\begin{figure}[H]
  \centering
  \includegraphics[width=140mm]{../img/kap02/intro/alignMotivationTemplate.png}
  \caption{Dvě sekundární RNA struktury s RNAcentral ID A) URS00006E712C, B)
  URS0000AB09C9 vygenerované nástrojem Traveler a C) vzorové sekundární RNA
  struktury d.5.e.S.oshimae.}
\end{figure}

Přidáním struktury, ze které jsou obě struktury odvozené, je mnohem jasnější,
na která místa koukat pří hledání podobností a rozdílů. Přesto nemusí být
nějaká podobnost hned jasná, a proto jsme se zaměřili na způsoby, jak znázornit
mapování nukleotidů na vzorovou strukturu.

\section{Důsledek využívaní vzorové struktury}

Využívání vzorové struktury pro analýzu má zřejmý a důležitý důsledek. Náš
nástroj usnadňuje analýzu $N$ struktur, které byly vygenerované ze stejné
vzorové struktury. 

Protože Traveler umožňuje zadat, která vzorová struktura se má použít pro
generování je teoreticky možné, porovnávat jakékoliv dvě sekundární RNA
struktury, pokud bude jejich rozložení vygenerované na základě stejné vzorové
struktury. Pokud se ale pokusíme vygenerovat rozložení struktury pomocí vzorové
struktury, která je úplně jiná, nebude existovat žádná podobnost se vzorovou
strukturou, a tím pádem naše metody, které spoléhají na podobnost vzorové
struktury s vygenerovanou nebudou užitečné.

\section{Překládání struktur}

Vygenerované sekundární struktury jsou podobné vzorové struktuře, a tím pádem
bývají podobné i ostatním vygenerovaným strukturám ze stejné vzorové struktury.
Dává proto smysl se pokusit struktury přes sebe přeložit, aby se spojily
společné části a vynikly rozdíly. Pouhým přeložením podobných struktur přes
sebe však získáme výsledek, který neposkytuje příliš zajímavé informace a je
málo přehledný.

\begin{figure}[H]
  \centering
  \includegraphics[width=140mm]{../img/kap02/align/structures.png}
  \caption{A) Struktura s RNAcentral ID URS00000B9D9D vygenerovaná nástrojem
  Traveler pomocí B) vzorové struktury d.5.b.A.madurae vedle sebe.}
\end{figure}

\begin{figure}[H]
  \centering
  \includegraphics[height=90mm]{../img/kap02/align/unaligned.png}
  \caption{A) Struktura s RNAcentral ID URS00000B9D9D vygenerovaná nástrojem
  Traveler pomocí B) vzorové struktury d.5.b.A.madurae přeložené přes sebe.}
\end{figure}

\subsection{Zarovnání}

Je potřeba najít způsob, jak řešit problém přesunu a zarovnání struktury,
protože manuální manipulace pomocí přetažení myší nebo zadávání pozice může být
zbytečně obtížná, zejména pokud se snažíme dosáhnout přesného zarovnání. Proto
je velmi užitečné umožnit zarovnání sekundární RNA struktury na konkrétní
nukleotid nebo skupinu nukleotidů ze vzorové struktury. 

Pokud je vybrán vzorový nukleotid $v$ pro nukleotid $n$ z ostatních struktur,
jehož vzor je $v$, je celá struktura přesunuta tak, aby se nukleotidy $v$ a $n$
překrývaly. 

\subsubsection{Skupiny nukleotidů}

Pokud se snažíme najít větší skupinu nukleotidů, na které chceme strukturu
zarovnat, koukáme se na posunutí pro konkrétní nukleotidy a následně vybereme
buď posunutí které zarovná nejvíc nukleotidů nebo to které zarovná skupinu
nukleotidů které jsou v části struktury, která nás právě zajímá.

Pro hledání konkrétního posunutí pro zarovnání jsme použili naivní způsob,
který nezaručuje nejlepší možné zarovnání, ale zároveň je poměrně přímočarý a
dává rozumné výsledky, díky již zmíněné podobnosti struktur.

Postupně se prochází každá struktura. V první iteraci se porovná pozice každého
nukleotidu s pozicí ve vzorové struktuře. Na základě pozice se roztřídí
nukleotidy. Třídění funguje tak, že do stejné skupiny indexované posunutím se
přidají všechny vzorové nukleotidy, na které se struktura zarovná daným posunutím.
Na konci první iterace se skupiny, které jsou menší než $x$ odeberou.

V druhé iteraci se postupuje podobně, ale používají se poslední vytvořené
skupiny na filtrování, tím se snažíme docílit, zarovnání na stejné nukleotidy.
Filtrování pomocí skupin ignoruje všechny nukleotidy, jejiž vzor není v nějaké
skupině, pomocí které filtrujeme. Pokud během iterace nevzniknou žádné skupiny,
řeší se iterace daná iterace zvlášť bez filtrování.

\begin{figure}[H]
  \centering
  \includegraphics[height=90mm]{../img/kap02/align/alignedAlpha1.png}
  \caption{A) Struktura s RNAcentral ID URS00000B9D9D vygenerovaná nástrojem
  Traveler pomocí B) vzorové struktury d.5.b.A.madurae přeložené přes sebe a
  zarovnané.}
\end{figure}

\subsection{Průhlednost struktur}

V přeložených a zarovnaných strukturách nelze snadno rozeznat, které nukleotidy
jsou společné a překrývají se a které nejsou. Přidáním průhlednosti je možné
tuto situaci rozlišit, protože překrývající se nukleotidy budou mít sytější
barvu než ty, které se nepřekrývají.

\begin{figure}[H]
  \centering
  \includegraphics[height=90mm]{../img/kap02/align/aligned.png}
  \caption{A) Struktura s RNAcentral ID URS00000B9D9D vygenerovaná nástrojem
  Traveler pomocí B) vzorové struktury d.5.b.A.madurae přeložené přes sebe,
  zarovnané a s průhledností.}
\end{figure}

\subsection{Rozmazání struktur}

Zarovnávání struktur bohužel neřeší všechny výzvy. Výsledné obrázky se mohou
zdát rozmazané. To je způsobeno tím, že ačkoli má nukleotid vzorový nukleotid,
který je stejný, jeho pozice se může v rozložení mírně lišit v důsledku metody
generování dat. Tento fakt může způsobit, že popisky nukleotidů vypadají
rozmazaně.

Jako přímočaré řešení se může zdát posunutí jednotlivých nukleotidů, které jsou
blízko sebe, aby dokonale překrývaly jejich vzory. Věříme, že by to vyřešilo
zmíněný problém bez významné deformace struktury.

\section{Obarvení struktur}

Vstupní data obsahují barevné označení nukleotidů. Slouží k lepšímu
zorientování se ve struktuře vzhledem ke vzorové struktuře. Jejich význam je
následující.

Černá barva značí, že nukleotid leží na poloze vzorového nukleotidu se stejným
názvem. Zelenou barvou jsou označený ty nukleotidy jejíž vzorový nukleotid bylo
třeba přejmenovat. Modrou barvou jsou vyznačený posunutý nukleotid. A poslední
růžovou barvu mají nově přidané nukleotidy.

\begin{figure}[H]
  \centering
  \includegraphics[width=140mm]{../img/kap03/inputDataColors.png}
  \caption{A) Struktura s RNAcentral ID URS00000B9D9D vygenerovaná nástrojem
  Traveler pomocí B) vzorové struktury d.5.b.A.madurae.}
\end{figure}

\section{Transformace na vzor}

Užitečnou metodou je transformace mezi vzorovou a cílovou strukturou. Každý
nukleotid, který má svůj vzorový nukleotid, se přemístí na pozici vzorového
nukleotidu a nukleotidy, které ve vzoru nejsou, jsou skryté. Tato metoda je
velmi užitečná pro práci s dvěma strukturami, které jsou si podobné, nebo pro
získání počátečního přehledu o tom, co je na co namapováno. 

Slabou stránkou této metody je její použití při práci s více než dvěma
strukturami nebo strukturami, které jsou velmi odlišné. V takových situacích se
na obrazovce děje mnoho věcí a je obtížné se soustředit a vypozorovat něco
užitečného.

\begin{figure}[H]
  \centering
  \includegraphics[width=140mm]{../img/kap02/animation.png}
  \caption{Struktura s RNAcentral ID URS00000B9D9D vygenerovaná nástrojem
  Traveler pomocí vzorové struktury d.5.b.A.madurae přeložené přes sebe A) před
  transformací a B) po transformaci.}
\end{figure}

\section{Mapovací čáry}

Vědomost o tom, který nukleotid se na co mapuje, může být důležitá pro odhalení
rozdílů a podobností mezi strukturami. V našem úsilí zprostředkovat tuto
informaci již před transformací na vzorovou sturkturu jsme přišli s čárami,
které spojují každý nukleotid s jeho vzorovým nukleotidem.

\begin{figure}[H]
  \centering
  \includegraphics[height=90mm]{../img/kap02/mappingLines/small.png}
  \caption{Struktura s RNAcentral ID URS00000B9D9D vygenerovaná nástrojem
  Traveler pomocí vzorové struktury d.5.b.A.madurae přeložené přes sebe s
  mapovacíma čárama.}
\end{figure}

Bohužel tento způsob se zvětšující se velikostí struktury stává velmi
nepřehledným, přesto si myslíme že můžou být užitečné.

\begin{figure}[H]
  \centering
  \includegraphics[height=100mm]{../img/kap02/mappingLines/big.png}
  \caption{Výřez z mnoha struktur vygenerovaných nástrojem Traveler pomocí vzorové struktury DD\_28S\_3D přeložených přes sebe s mapovacími čárami.
  Každá struktura má vlastní barvu mapovacích čar.}
\end{figure}

\section{Využití stromu}

V projektu Traveler byla použita stromová reprezentace pro sekundární RNA
strukturu, ve které je vnitřní vrchol, tedy má více jak jednu hranu, bázový pár
a listem, tedy má jednu nebo žádnou hranu, je nespárovaný nukleotid. Strom pak
lze vytvořit následovně. První nepárové nukleotidy ze začátku a z konce přidáme
jako samostatné vrcholy. Potom jdeme postupně po bázových párech. První bázový
pár tvoří kořen. Další bázové páry připojujeme a tvoříme strom. Pokud narazíme
na nespárovaný nukleotid přidáme ho jako list k poslednímu přidanému bázovému
páru. Větvení struktury vyústí ve větvení stromu. 

\begin{figure}[H]
  \centering
  \includegraphics[width=140mm]{../img/kap02/tree/tree.png}
  \caption{Vlevo je uměle vytvořená struktura zobrazená nástrojem Varna. Vpravo je jeho stromová reprezentace.}
\end{figure}

Stromovu strukturu jsme neměli v úmyslu použít na nic konkrétního, ale chtěli
jsme prozkoumat možnosti lokálních transformací struktury pro dosažení
zarovnání více podobných částí, které jsou třeba jenom posunuté, jako je to
například v následující části dvou struktur, které jsou zarovnané dvěma
způsoby.

\begin{figure}[H]
  \centering
  \includegraphics[height=100mm]{../img/kap02/tree/align1.png}
  \caption{První způsob zarovnání.}
\end{figure}

\begin{figure}[H]
  \centering
  \includegraphics[height=100mm]{../img/kap02/tree/align2.png}
  \caption{Druhý způsob zarovnání.}
\end{figure}

Po zvážení jsme dospěli k názoru, že není možné provést úpravu struktury, která
by významně nedeformovala strukturu. Jakákoli úprava části struktury znamená
posunutí nějakého zbytku struktury, což může zasáhnout do jiné části. Proto by
bylo nezbytné odebrat některé nukleotidy. Ty, které nejvíce překážejí, jsou ty
přidané. Pokud bychom odebrali tyto přidané nukleotidy, dostali bychom se na
stejný výsledek jako s transformací struktury na vzorovou strukturu.

\section{Zvolené metody}

Naše knihovna přímo podporuje překládání a zarovnání struktur na konkrétní
nukleotid nebo skupinu nukleotidů bez zarovnání na úrovni jednotlivých
nukleotidů pro odstranění rozmazání struktur. Lze transformovat struktury na
vzorovou strukturu a upravovat průhlednost struktur.


\chapter{Programátorská dokumentace}

\section{Vstupní data}

\begin{figure}[H]
  \centering
  \includegraphics[width=145mm]{../img/rnaInput.png}
  \caption{Interface pro vstupní data}
\end{figure}

\begin{figure}[H]
  \centering
  \includegraphics[width=145mm]{../img/rnaVis.png}
  \caption{Interface pro vstupní data}
\end{figure}

\begin{figure}[H]
  \centering
  \includegraphics[width=145mm]{../img/dataContainer.png}
  \caption{Interface pro vstupní data}
\end{figure}

\begin{figure}[H]
  \centering
  \includegraphics[width=145mm]{../img/translationGroups.png}
  \caption{Interface pro vstupní translationGroups}
\end{figure}

\begin{figure}[H]
  \centering
  \includegraphics[width=145mm]{../img/title.png}
  \caption{Interface pro vstupní data}
\end{figure}

\begin{figure}[H]
  \centering
  \includegraphics[width=145mm]{../img/styles.png}
  \caption{Interface pro vstupní data}
\end{figure}

\begin{figure}[H]
  \centering
  \includegraphics[width=145mm]{../img/primitives.png}
  \caption{Interface pro vstupní data}
\end{figure}

\begin{figure}[H]
  \centering
  \includegraphics[width=145mm]{../img/iTransformation.png}
  \caption{Interface pro vstupní data}
\end{figure}

\begin{figure}[H]
  \centering
  \includegraphics[width=145mm]{../img/draw.png}
  \caption{Interface pro vstupní data}
\end{figure}

\begin{figure}[H]
  \centering
  \includegraphics[width=145mm]{../img/containerFactory.png}
  \caption{Interface pro vstupní data}
\end{figure}

\begin{figure}[H]
  \centering
  \includegraphics[width=145mm]{../img/animations.png}
  \caption{Interface pro vstupní data}
\end{figure}




\chapter*{Závěr}
\addcontentsline{toc}{chapter}{Závěr}

Cílem této práce bylo vytvořit knihovnu pro web, pomocí které bude možné
vytvářet vizualizace několika sekundárních struktur RNA a porovnávat je mezi
sebou. V této práci jsme na začátek čtenáře krátce seznámili s RNA, její
funkcí\ref{rnauvod} a vizualizací. Následně jsme představili knihovnu napsanou
v jazyce Typescript, která pomocí jednoduchého rozhraní\ref{seznameni}
zpřístupňuje metody pro porovnání dostatečné podobných struktur\ref{porovnani}.
Knihovna nabízí plynulé vykreslování několika struktur najednou\ref{canvas} a
je jednoduše přístupná přes npm nebo GitHub\ref{seznameni}.

Ukázalo se, že vizualizace několika sekundárních struktur RNA je složitá.
Dokázali jsme vytvořit knihovnu pro vizualizaci, ale použité metody jsou vhodné
pouze pro dvě nebo tři struktury najednou, při větším množství se metody
neukázaly nápomocné, především protože vizualizace přestane být přehledná.
Navíc jsme se omezili na struktury, které jsou si dostatečně podobné.

V budoucnu by bylo užitečné knihovnu rozšířit o metody, které dokážou pracovat
s libovolnýma strukturami nebo se strukturami, které patří do stejné
klasifikace\cite{rfam}. Tím bychom se přiblížili k původnímu cíli této práce.
Dalším vylepšením by mohla být optimalizace kódu, aby byla knihovna schopná
pracovat plynule s větším množstvím struktur.


%%% Seznam použité literatury
\include{literatura}

%%% Obrázky v bakalářské práci
%%% (pokud jich je malé množství, obvykle není třeba seznam uvádět)
\listoffigures

% \lstlistoflistings

%%% Tabulky v bakalářské práci (opět nemusí být nutné uvádět)
%%% U matematických prací může být lepší přemístit seznam tabulek na začátek práce.
% \listoftables

%%% Použité zkratky v bakalářské práci (opět nemusí být nutné uvádět)
%%% U matematických prací může být lepší přemístit seznam zkratek na začátek práce.
% \chapwithtoc{Seznam použitých zkratek}

%%% Přílohy k bakalářské práci, existují-li. Každá příloha musí být alespoň jednou
%%% odkazována z vlastního textu práce. Přílohy se číslují.
%%%
%%% Do tištěné verze se spíše hodí přílohy, které lze číst a prohlížet (dodatečné
%%% tabulky a grafy, různé textové doplňky, ukázky výstupů z počítačových programů,
%%% apod.). Do elektronické verze se hodí přílohy, které budou spíše používány
%%% v elektronické podobě než čteny (zdrojové kódy programů, datové soubory,
%%% interaktivní grafy apod.). Elektronické přílohy se nahrávají do SISu a lze
%%% je také do práce vložit na CD/DVD. Povolené formáty souborů specifikuje
%%% opatření rektora č. 72/2017.
\appendix
\chapter{Přílohy}

\section{SQL dotazy pro získání testovacích dat}

\begin{lstlisting}[caption={SQL dotaz pro získání vzorových struktur},label=vzorsql]
  SELECT 
    sslm.id, 
    sslm.model_name, 
    sslm.model_source, 
    sslm.model_length, 
    sslm.rna_type, 
    COUNT(*)
  FROM 
    rnacen.rnc_secondary_structure_layout_models sslm 
    JOIN rnacen.rnc_secondary_structure_layout ssl 
    ON (ssl.model_id = sslm.id)
  GROUP BY
    sslm.id
  ORDER BY model_name
\end{lstlisting}

% \begin{lstlisting}[caption={Dotaz pro získání struktur vygenerovaných ze zvolených vzorových struktur},label=structsql]
%   SELECT 
%     sslm.id, 
%     sslm.model_name, 
%     sslm.model_source, 
%     sslm.model_length, 
%     sslm.rna_type, 
%     COUNT(*)
%   FROM 
%     rnacen.rnc_secondary_structure_layout_models sslm 
%     JOIN rnacen.rnc_secondary_structure_layout ssl 
%     ON (ssl.model_id = sslm.id)
%   GROUP BY
%     sslm.id
%   HAVING
%     sslm.model_name IN ('LD_SSU_3D'
%               , 'RF03163'
%               , 'RF03163'
%               , 'RF04183'
%               , 'RNAseP_e_H_sapiens_3D'
%               , 'EC_SSU_3D'
%               , 'TT_SSU_3D'
%               , 'mt_TetT_LSU_3D'
%               , 'E-Thr'
%               , 'mHS_LSU_3D'
%               , 'DM_LSU_3D'
%               , 'DD_28S_3D'
%               , 'RF02705'
%               , 'RF00025'
%               , 'd.5.b.Thermus.sp'
%               , 'd.16.b.S.gougerotii'
%               , 'd.5.e.S.oshimae'
%               , 'd.16.e.P.falciparum.S'
%               , 'a.16.c.C.ruber'
%               , 'HS_LSU_3D'
%               , 'TeT_LSU_3D'
%               )
%   ORDER BY model_name
% \end{lstlisting}

\begin{lstlisting}[caption={SQL dotaz pro získání struktur vygenerovaných ze zvolených vzorových struktur},label=structsql]
  SELECT 
    *,
    rna.seq_short
  FROM 
  (
    SELECT row_number() 
    OVER (PARTITION BY ssl.model_id ORDER BY ssl.id ASC) As rn,
      ssl.*,
      sslm.model_name as template_name, 
      sslm.model_source as template_source, 
      sslm.model_length as template_length, 
      sslm.rna_type
    FROM
      (SELECT 
         * 
       FROM 
         rnacen.rnc_secondary_structure_layout_models
      WHERE
        model_name IN ('LD_SSU_3D'
                , 'RF03163'
                , 'RF03163'
                , 'RF04183'
                , 'RNAseP_e_H_sapiens_3D'
                , 'EC_SSU_3D'
                , 'TT_SSU_3D'
                , 'mt_TetT_LSU_3D'
                , 'E-Thr'
                , 'mHS_LSU_3D'
                , 'DM_LSU_3D'
                , 'DD_28S_3D'
                , 'RF02705'
                , 'RF00025'
                , 'd.5.b.Thermus.sp'
                , 'd.16.b.S.gougerotii'
                , 'd.5.e.S.oshimae'
                , 'd.16.e.P.falciparum.S'
                , 'a.16.c.C.ruber'
                , 'HS_LSU_3D'
                , 'TeT_LSU_3D'
                )
    ) sslm
    JOIN rnacen.rnc_secondary_structure_layout ssl 
    ON ( sslm.id = ssl.model_id)        
      ORDER BY sslm.model_name    
  ) t
  JOIN rnacen.rna rna ON (rna.upi = t.urs)
  WHERE rn < 20
\end{lstlisting}



\openright
\end{document}
