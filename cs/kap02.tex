
\chapter{Metody vizualizace a porovnání}

Cílem této knihovny je nejen vizualizovat sekundární RNA strukturu, ale
především zjednodušit analýzu rozdílů a podobností vícero RNA struktur. Právě
proto jsme se zaměřili na práci s radial diagramem.

Zároveň jsme viděli potenciál v generování rozložení na základě vzorové
struktury, jako to děla nástroj Traveler. Výstupem Traveleru je soubor ve
formátu JSON, který obsahuje mimo jiné informaci o vzoru každého nukletidu a i
informaci o provedených editacích.

Rozhodli jsme se proto v našich metodách využívat právě výše zmíněného mapování
na vzorovou strukturu. Náš nástroj tím pádem je schopný pracovat se
strukturama, jejiž rozložení je vygenerované na základě stejné vzorové
struktury.

Jednou z metod je transformace z a na vzorovou strukturu. Každý nukleotid,
který má vzorový nukleotid se přemístí na pozici vzorového nukleotidu a ty
nukleotidy, které vzor nemají jsou schovaný. Metoda je velmi příjemná pro práci
se dvěma strukturama, které si jsou podobné nebo pro počáteční přehled co je na
co namapované. Slabá stránka této metody je zjevná při práci s vícero
strukturami nebo strukturami, které jsou velmi odlišné. V takových situacích se
toho na displeji děje hodně a je složité se soustředit a vypozorovat něco
užitečného.

Vědět který nukleotid se na co mapuje může být velmi užitečné pro odhalení
rozdílů a podobností struktur. Snažili jsme se najít další způosob, jak tuto
informaci předat, jediné s čím jsme přišli jsou čáry, které spojují nukletid se
vzorovým. Bohužel tento způsob se zvětšující se velikostí struktury stáva velmi
nepřehledným, přesto si myslíme že můžou být užitečné a v naše knihovna je
podporuje.

Protože jsou struktury odvozené od stejného vzoru jsou typicky velmi podobné,
dává proto smysl mít možnost je přeložit přes sebe, aby splynuli společné části
a vynikly ty rozdílné. Manipulací se strukturou ručně ať už přetažením myši
nebo zadáním pozice může být zbytečně otravné především kvůli zarovnání. Přijde
nám proto velmi užitečné mít možnost zarovnat sekundární RNA strukturu na
konkrétní nukleotid nebo skupinu nukleotidů. Obojího lze s naší knihovnou
pohodlně dosáhnout, včetně nalezení posunutí, kterým lze zarovnat skupiny
nukletidů

Zarovnávání struktur bohužel nedá vždy na první pohled očekávaný výsledek,
protože ačkoli má nukleotid vzorový nukleotid, od kterého se nijak neliší může
stále jeho pozice být mírně posunutá. Je to dáno metodou generování dat.
Popisky nukleotidů můžou tím pádem vypadat trochu rozmazaně. Jako přímočaré
řešení by se mohlo zdát posunout jednotilvé nukleotidy, které jsou blízko, aby
dokonale překrývali jejich vzor. Věříme, že by to vyřešilo zmíněný problém,
nicméně naše knihovna tuto funkci nijak přímo neimplementuje.

V rámci naší knihovny vznikla i webová
aplikace\footnote{https://michalhercik.github.io/rna-visualizer/}, která
demonstruje možnosti naší knihovny. Umožňuje pracovat vždy jen s jednou metodou
nebo se všemi metodami usnadňující porovnání dvou sekundárních struktur RNA,
které jsou zmíněné v této kapitole.
